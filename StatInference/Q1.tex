\documentclass[]{article}
\usepackage{lmodern}
\usepackage{amssymb,amsmath}
\usepackage{ifxetex,ifluatex}
\usepackage{fixltx2e} % provides \textsubscript
\ifnum 0\ifxetex 1\fi\ifluatex 1\fi=0 % if pdftex
  \usepackage[T1]{fontenc}
  \usepackage[utf8]{inputenc}
\else % if luatex or xelatex
  \ifxetex
    \usepackage{mathspec}
  \else
    \usepackage{fontspec}
  \fi
  \defaultfontfeatures{Ligatures=TeX,Scale=MatchLowercase}
\fi
% use upquote if available, for straight quotes in verbatim environments
\IfFileExists{upquote.sty}{\usepackage{upquote}}{}
% use microtype if available
\IfFileExists{microtype.sty}{%
\usepackage{microtype}
\UseMicrotypeSet[protrusion]{basicmath} % disable protrusion for tt fonts
}{}
\usepackage[margin=1in]{geometry}
\usepackage{hyperref}
\hypersetup{unicode=true,
            pdftitle={Statistical Inference - Simulation Project},
            pdfauthor={Joe DeMaro},
            pdfborder={0 0 0},
            breaklinks=true}
\urlstyle{same}  % don't use monospace font for urls
\usepackage{color}
\usepackage{fancyvrb}
\newcommand{\VerbBar}{|}
\newcommand{\VERB}{\Verb[commandchars=\\\{\}]}
\DefineVerbatimEnvironment{Highlighting}{Verbatim}{commandchars=\\\{\}}
% Add ',fontsize=\small' for more characters per line
\usepackage{framed}
\definecolor{shadecolor}{RGB}{248,248,248}
\newenvironment{Shaded}{\begin{snugshade}}{\end{snugshade}}
\newcommand{\KeywordTok}[1]{\textcolor[rgb]{0.13,0.29,0.53}{\textbf{#1}}}
\newcommand{\DataTypeTok}[1]{\textcolor[rgb]{0.13,0.29,0.53}{#1}}
\newcommand{\DecValTok}[1]{\textcolor[rgb]{0.00,0.00,0.81}{#1}}
\newcommand{\BaseNTok}[1]{\textcolor[rgb]{0.00,0.00,0.81}{#1}}
\newcommand{\FloatTok}[1]{\textcolor[rgb]{0.00,0.00,0.81}{#1}}
\newcommand{\ConstantTok}[1]{\textcolor[rgb]{0.00,0.00,0.00}{#1}}
\newcommand{\CharTok}[1]{\textcolor[rgb]{0.31,0.60,0.02}{#1}}
\newcommand{\SpecialCharTok}[1]{\textcolor[rgb]{0.00,0.00,0.00}{#1}}
\newcommand{\StringTok}[1]{\textcolor[rgb]{0.31,0.60,0.02}{#1}}
\newcommand{\VerbatimStringTok}[1]{\textcolor[rgb]{0.31,0.60,0.02}{#1}}
\newcommand{\SpecialStringTok}[1]{\textcolor[rgb]{0.31,0.60,0.02}{#1}}
\newcommand{\ImportTok}[1]{#1}
\newcommand{\CommentTok}[1]{\textcolor[rgb]{0.56,0.35,0.01}{\textit{#1}}}
\newcommand{\DocumentationTok}[1]{\textcolor[rgb]{0.56,0.35,0.01}{\textbf{\textit{#1}}}}
\newcommand{\AnnotationTok}[1]{\textcolor[rgb]{0.56,0.35,0.01}{\textbf{\textit{#1}}}}
\newcommand{\CommentVarTok}[1]{\textcolor[rgb]{0.56,0.35,0.01}{\textbf{\textit{#1}}}}
\newcommand{\OtherTok}[1]{\textcolor[rgb]{0.56,0.35,0.01}{#1}}
\newcommand{\FunctionTok}[1]{\textcolor[rgb]{0.00,0.00,0.00}{#1}}
\newcommand{\VariableTok}[1]{\textcolor[rgb]{0.00,0.00,0.00}{#1}}
\newcommand{\ControlFlowTok}[1]{\textcolor[rgb]{0.13,0.29,0.53}{\textbf{#1}}}
\newcommand{\OperatorTok}[1]{\textcolor[rgb]{0.81,0.36,0.00}{\textbf{#1}}}
\newcommand{\BuiltInTok}[1]{#1}
\newcommand{\ExtensionTok}[1]{#1}
\newcommand{\PreprocessorTok}[1]{\textcolor[rgb]{0.56,0.35,0.01}{\textit{#1}}}
\newcommand{\AttributeTok}[1]{\textcolor[rgb]{0.77,0.63,0.00}{#1}}
\newcommand{\RegionMarkerTok}[1]{#1}
\newcommand{\InformationTok}[1]{\textcolor[rgb]{0.56,0.35,0.01}{\textbf{\textit{#1}}}}
\newcommand{\WarningTok}[1]{\textcolor[rgb]{0.56,0.35,0.01}{\textbf{\textit{#1}}}}
\newcommand{\AlertTok}[1]{\textcolor[rgb]{0.94,0.16,0.16}{#1}}
\newcommand{\ErrorTok}[1]{\textcolor[rgb]{0.64,0.00,0.00}{\textbf{#1}}}
\newcommand{\NormalTok}[1]{#1}
\usepackage{longtable,booktabs}
\usepackage{graphicx,grffile}
\makeatletter
\def\maxwidth{\ifdim\Gin@nat@width>\linewidth\linewidth\else\Gin@nat@width\fi}
\def\maxheight{\ifdim\Gin@nat@height>\textheight\textheight\else\Gin@nat@height\fi}
\makeatother
% Scale images if necessary, so that they will not overflow the page
% margins by default, and it is still possible to overwrite the defaults
% using explicit options in \includegraphics[width, height, ...]{}
\setkeys{Gin}{width=\maxwidth,height=\maxheight,keepaspectratio}
\IfFileExists{parskip.sty}{%
\usepackage{parskip}
}{% else
\setlength{\parindent}{0pt}
\setlength{\parskip}{6pt plus 2pt minus 1pt}
}
\setlength{\emergencystretch}{3em}  % prevent overfull lines
\providecommand{\tightlist}{%
  \setlength{\itemsep}{0pt}\setlength{\parskip}{0pt}}
\setcounter{secnumdepth}{0}
% Redefines (sub)paragraphs to behave more like sections
\ifx\paragraph\undefined\else
\let\oldparagraph\paragraph
\renewcommand{\paragraph}[1]{\oldparagraph{#1}\mbox{}}
\fi
\ifx\subparagraph\undefined\else
\let\oldsubparagraph\subparagraph
\renewcommand{\subparagraph}[1]{\oldsubparagraph{#1}\mbox{}}
\fi

%%% Use protect on footnotes to avoid problems with footnotes in titles
\let\rmarkdownfootnote\footnote%
\def\footnote{\protect\rmarkdownfootnote}

%%% Change title format to be more compact
\usepackage{titling}

% Create subtitle command for use in maketitle
\newcommand{\subtitle}[1]{
  \posttitle{
    \begin{center}\large#1\end{center}
    }
}

\setlength{\droptitle}{-2em}

  \title{Statistical Inference - Simulation Project}
    \pretitle{\vspace{\droptitle}\centering\huge}
  \posttitle{\par}
    \author{Joe DeMaro}
    \preauthor{\centering\large\emph}
  \postauthor{\par}
      \predate{\centering\large\emph}
  \postdate{\par}
    \date{December 2018}


\begin{document}
\maketitle

\subsection{Set Parameters}\label{set-parameters}

\begin{Shaded}
\begin{Highlighting}[]
\KeywordTok{setwd}\NormalTok{(}\StringTok{"c:}\CharTok{\textbackslash{}\textbackslash{}}\StringTok{rprograms}\CharTok{\textbackslash{}\textbackslash{}}\StringTok{statinference"}\NormalTok{)}
\NormalTok{n <-}\StringTok{ }\DecValTok{1000}
\NormalTok{group <-}\StringTok{ }\DecValTok{40}
\NormalTok{lambda <-}\StringTok{ }\FloatTok{0.2}
\end{Highlighting}
\end{Shaded}

\subsection{Calculate Theoretical Mean and
SD}\label{calculate-theoretical-mean-and-sd}

\begin{Shaded}
\begin{Highlighting}[]
\NormalTok{theoreticalmean <-}\StringTok{ }\DecValTok{1} \OperatorTok{/}\StringTok{ }\NormalTok{lambda}
\NormalTok{theoreticalsd <-}\StringTok{ }\NormalTok{(}\DecValTok{1} \OperatorTok{/}\StringTok{ }\NormalTok{(lambda }\OperatorTok{*}\StringTok{ }\NormalTok{lambda)) }\OperatorTok{/}\StringTok{ }\NormalTok{group}
\end{Highlighting}
\end{Shaded}

\subsection{Run Simulation}\label{run-simulation}

\begin{Shaded}
\begin{Highlighting}[]
\CommentTok{# run the simulation}
\NormalTok{data  <-}\StringTok{ }\KeywordTok{rexp}\NormalTok{(n }\OperatorTok{*}\StringTok{ }\NormalTok{group, lambda )}
\NormalTok{matrixdata <-}\StringTok{ }\KeywordTok{matrix}\NormalTok{(data, n, group)}
\NormalTok{matrixmean <-}\StringTok{ }\KeywordTok{apply}\NormalTok{(matrixdata,}\DecValTok{1}\NormalTok{,mean)}
\end{Highlighting}
\end{Shaded}

\subsection{Tabulate Simulation Data}\label{tabulate-simulation-data}

\begin{Shaded}
\begin{Highlighting}[]
\NormalTok{simmean <-}\StringTok{ }\KeywordTok{mean}\NormalTok{(matrixmean)}
\NormalTok{simsd <-}\StringTok{ }\KeywordTok{sd}\NormalTok{(matrixmean)}
\NormalTok{simvar <-}\StringTok{ }\KeywordTok{var}\NormalTok{(matrixmean)}

\CommentTok{#compare to CLT}
\NormalTok{simse =}\StringTok{ }\NormalTok{simsd }\OperatorTok{/}\StringTok{ }\KeywordTok{sqrt}\NormalTok{(group)}
\CommentTok{#calculate confidence}
\NormalTok{low <-}\StringTok{ }\NormalTok{simmean }\OperatorTok{-}\StringTok{ }\FloatTok{1.96} \OperatorTok{*}\StringTok{ }\NormalTok{simse}
\NormalTok{high <-}\StringTok{ }\NormalTok{simmean }\OperatorTok{+}\StringTok{ }\FloatTok{1.96} \OperatorTok{*}\StringTok{ }\NormalTok{simse}

\KeywordTok{library}\NormalTok{(knitr)}

\NormalTok{theorecticalData <-}\StringTok{ }\KeywordTok{c}\NormalTok{(theoreticalmean,theoreticalsd)}

\NormalTok{simulationdata <-}\StringTok{ }\KeywordTok{c}\NormalTok{(simmean, simsd)}

\NormalTok{df =}\StringTok{ }\KeywordTok{data.frame}\NormalTok{(theorecticalData,simulationdata)}
\KeywordTok{colnames}\NormalTok{(df) <-}\StringTok{ }\KeywordTok{c}\NormalTok{(}\StringTok{"Theorectical"}\NormalTok{, }\StringTok{"Simulated"}\NormalTok{)}
\KeywordTok{row.names}\NormalTok{(df) <-}\StringTok{ }\KeywordTok{c}\NormalTok{(}\StringTok{"Mean"}\NormalTok{, }\StringTok{"SD"}\NormalTok{)}

\KeywordTok{kable}\NormalTok{(df, }\DataTypeTok{format=}\StringTok{"markdown"}\NormalTok{, }\DataTypeTok{caption=}\StringTok{"Comparison between Theorecital and Simulated Data"}\NormalTok{)}
\end{Highlighting}
\end{Shaded}

\begin{longtable}[]{@{}lrr@{}}
\toprule
& Theorectical & Simulated\tabularnewline
\midrule
\endhead
Mean & 5.000 & 5.0307701\tabularnewline
SD & 0.625 & 0.8153677\tabularnewline
\bottomrule
\end{longtable}

\subsection{Distribution of Simulated Data around the mean (95\%
CI)}\label{distribution-of-simulated-data-around-the-mean-95-ci}

\begin{Shaded}
\begin{Highlighting}[]
\NormalTok{ciData <-}\StringTok{ }\KeywordTok{c}\NormalTok{(low, high)}
\NormalTok{ciDF <-}\StringTok{ }\KeywordTok{data.frame}\NormalTok{(ciData)}
\KeywordTok{row.names}\NormalTok{(ciDF) <-}\StringTok{ }\KeywordTok{c}\NormalTok{(}\StringTok{"Low"}\NormalTok{, }\StringTok{"High"}\NormalTok{)}
\KeywordTok{kable}\NormalTok{(ciDF, }\DataTypeTok{format=}\StringTok{"markdown"}\NormalTok{, }\DataTypeTok{caption=}\StringTok{"CI for Simulated Data"}\NormalTok{, }\DataTypeTok{title=}\StringTok{"Simulated Data CI"}\NormalTok{)}
\end{Highlighting}
\end{Shaded}

\begin{longtable}[]{@{}lr@{}}
\toprule
& ciData\tabularnewline
\midrule
\endhead
Low & 4.778085\tabularnewline
High & 5.283455\tabularnewline
\bottomrule
\end{longtable}

\subsection{Histogram of Means from
Simulation}\label{histogram-of-means-from-simulation}

\begin{Shaded}
\begin{Highlighting}[]
\KeywordTok{hist}\NormalTok{(matrixmean)}
\end{Highlighting}
\end{Shaded}

\includegraphics{Q1_files/figure-latex/unnamed-chunk-6-1.pdf}

\subsection{Analysis}\label{analysis}

\subsubsection{Simulation data provides us with available means to under
the concepts of population and sample data in regards to Normal
Distribution, the leader of all distributions. After generating and
tabulating the sample data, comparison clearly show that rejecting the
null hypothesis is the appropriate action in this
experiment.}\label{simulation-data-provides-us-with-available-means-to-under-the-concepts-of-population-and-sample-data-in-regards-to-normal-distribution-the-leader-of-all-distributions.-after-generating-and-tabulating-the-sample-data-comparison-clearly-show-that-rejecting-the-null-hypothesis-is-the-appropriate-action-in-this-experiment.}


\end{document}
